\section{Условие}
Родительский процесс создает дочерний процесс. Первой строкой пользователь в консоль
родительского процесса вводит имя файла, которое будет использовано для открытия File с таким
именем на запись. Родительский и дочерний процесс должны быть представлены разными программами.
Родительский процесс принимает от пользователя строки произвольной длины и пересылает их через
memory-mapped files. Процесс child проверяет строки на валидность правилу. Если строка соответствует правилу,
то она выводится в файл, иначе через memory-mapped files выводится
информация об ошибке. Родительский процесс полученные от child ошибки выводит в
стандартный поток вывода.


{\bfseries Цель работы:}
Приобретение практических навыков в:
\begin{itemize}
    \item Освоение принципов работы с файловыми системами
	\item Обеспечение обмена данных между процессами посредством технологии «File mapping»
\end{itemize}

{\bfseries Задание:}
Разработать программу, состоящую из двух процессов — родительского и дочернего, взаимодействующих через отображаемые файлы (memory-mapped files).

Родительский процесс должен:
\begin{itemize}
    \item Запрашивать у пользователя имя файла и передавать его дочернему процессу через командную строку;
    \item Принимать от пользователя строки с числами и передавать их дочернему процессу через разделяемую память;
    \item Ожидать обработки данных дочерним процессом через механизм синхронизации в разделяемой памяти;
    \item Завершать работу при получении сигнала об ошибке деления на ноль через разделяемую память.
\end{itemize}

Дочерний процесс должен:
\begin{itemize}
    \item Получить от родительского процесса имя файла через аргументы командной строки и открыть его для записи;
    \item Принимать строки с числами от родительского процесса через разделяемую память;
    \item Выполнять операцию последовательного деления первого числа на последующие (кумулятивное деление);
    \item Записывать результаты вычислений в файл;
    \item Проверять деление на ноль и при обнаружении устанавливать флаг ошибки в разделяемой памяти;
    \item Завершать работу при обнаружении деления на ноль с аварийным кодом возврата.
\end{itemize}

{\bfseries Вариант:} 3


