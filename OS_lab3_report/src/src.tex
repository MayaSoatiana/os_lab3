\section{Метод решения}

Для решения задачи применена архитектура с двумя процессами (родительским и дочерним), взаимодействующими через технологию File Mapping (отображаемые файлы). Родительский процесс отвечает за взаимодействие с пользователем и координацию работы, тогда как дочерний процесс выполняет вычислительную логику и работу с файлами.

\subsection{Основной алгоритм работы}
\begin{enumerate}
    \item \textbf{Инициализация:} Родительский процесс создает разделяемую память (shared memory) и запрашивает у пользователя имя файла для сохранения результатов
    \item \textbf{Запуск процесса:} Создание дочернего процесса с передачей имени файла через аргументы командной строки
    \item \textbf{Синхронизация процессов:} Оба процесса подключаются к общей области памяти для обмена данными
    \item \textbf{Обработка данных:}
    \begin{itemize}
        \item Родительский процесс читает числовые данные от пользователя и записывает их в разделяемую память
        \item Дочерний процесс отслеживает флаг data\_ready, читает числа из разделяемой памяти
        \item Дочерний процесс выполняет кумулятивное деление первого числа на последующие с проверкой на деление на ноль
        \item Результаты вычислений записываются в файл, устанавливается флаг processing\_complete
    \end{itemize}
    \item \textbf{Завершение работы:} Корректное закрытие процессов при получении команды exit или ошибке деления на ноль
\end{enumerate}

\subsection{Особенности реализации}
Для обеспечения межпроцессного взаимодействия использована технология File Mapping Windows API. Реализована синхронизация процессов через флаги в разделяемой памяти. Обработка ошибок включает проверку системных вызовов, валидацию входных данных и аварийное завершение при делении на ноль.

\section{Описание программы}

Программа реализована в модульном стиле и состоит из четырех основных компонентов.

\subsection{Модуль parent.c}
Реализует логику родительского процесса:
\begin{itemize}
    \item Создание и управление дочерним процессом через функции библиотеки
    \item Создание и инициализация разделяемой памяти для межпроцессного взаимодействия
    \item Взаимодействие с пользователем (ввод имени файла и числовых данных)
    \item Запись данных в разделяемую память и ожидание обработки дочерним процессом
    \item Обработка флага division\_by\_zero для аварийного завершения работы
    \item Координация корректного завершения работы процессов
\end{itemize}

\subsection{Модуль child.c}
Содержит вычислительную логику и работу с файлами:
\begin{itemize}
    \item Получение имени файла через аргументы командной строки
    \item Подключение к существующей разделяемой памяти
    \item Отслеживание флага data\_ready для получения новых данных
    \item Парсинг числовых данных и валидация входных значений
    \item Выполнение кумулятивного деления чисел с проверкой на деление на ноль
    \item Запись промежуточных и финальных результатов в файл
    \item Установка флагов состояния в разделяемой памяти
\end{itemize}

\subsection{Модуль library.h/c}
Предоставляет кроссплатформенные абстракции и утилиты:
\begin{itemize}
    \item \textbf{Структуры данных:} process\_t (для управления процессами)
    \item \textbf{Функции управления процессами:} CpProcessCreate, CpProcessWrite, CpProcessRead, CpProcessClose
    \item \textbf{Функции работы с разделяемой памятью:} CpCreateSharedMemory, CpOpenSharedMemory, CpCloseSharedMemory
    \item \textbf{Вспомогательные функции:} TrimNewline - удаление символов новой строки
    \item \textbf{Кроссплатформенные обертки:} унифицированный API для Windows и Unix систем
\end{itemize}

\subsection{Модуль shared\_data.h}
Определяет структуру данных для разделяемой памяти:
\begin{itemize}
    \item filename - имя файла для записи результатов
    \item numbers - массив чисел для обработки
    \item numbers\_count - количество чисел в массиве
    \item data\_ready - флаг готовности данных для обработки
    \item processing\_complete - флаг завершения обработки
    \item division\_by\_zero - флаг ошибки деления на ноль
    \item terminate\_process - флаг запроса на завершение работы
\end{itemize}

\subsection{Используемые системные вызовы}
\begin{itemize}
    \item \textbf{Windows API:} CreateFileMapping, MapViewOfFile, OpenFileMapping, UnmapViewOfFile
    \item \textbf{Управление процессами:} CreateProcess, CreatePipe, ReadFile, WriteFile, CloseHandle, WaitForSingleObject
    \item \textbf{Файловые операции:} fopen, fclose, fprintf, fflush
    \item \textbf{Обработка строк:} strtok, strncpy, atoi, strlen
\end{itemize}

Архитектура программы обеспечивает четкое разделение ответственности между процессами и использует современные механизмы межпроцессного взаимодействия через технологию File Mapping.