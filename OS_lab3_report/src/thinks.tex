\section{Выводы}

В ходе выполнения лабораторной работы были успешно достигнуты поставленные цели:

\begin{enumerate}
    \item \textbf{Освоены принципы работы с файловыми системами:} На практике применены операции создания, открытия и записи в файлы. Реализована работа с файловой системой через стандартные функции языка Си (\texttt{fopen()}, \texttt{fprintf()}, \texttt{fclose()}) для сохранения результатов вычислений
    
    \item \textbf{Обеспечен обмен данных между процессами посредством технологии File Mapping:} Организовано межпроцессное взаимодействие через разделяемую память с использованием Windows API (\texttt{CreateFileMapping()}, \texttt{MapViewOfFile()}). Реализована синхронизация процессов через флаги в общей структуре данных
    
    \item \textbf{Создана программа для работы с процессами:} Разработана архитектура с родительским и дочерним процессами, где родительский процесс создает дочерний и координирует его работу через механизмы разделяемой памяти
    
    \item \textbf{Реализована обработка системных ошибок:} 
    \begin{itemize}
        \item Обработка ошибок создания и отображения разделяемой памяти
        \item Контроль ошибок открытия и записи в файлы
        \item Обнаружение и обработка арифметической ошибки деления на ноль
        \item Корректное освобождение ресурсов при аварийном завершении
    \end{itemize}
\end{enumerate}

Работа продемонстрировала эффективность использования технологии File Mapping для высокоскоростного обмена данными между процессами. Полученный опыт может быть применен при разработке сложных многопроцессных приложений, требующих интенсивного обмена данными.

\pagebreak